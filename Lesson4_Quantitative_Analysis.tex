\PassOptionsToPackage{unicode=true}{hyperref} % options for packages loaded elsewhere
\PassOptionsToPackage{hyphens}{url}
%
\documentclass[
]{article}
\usepackage{lmodern}
\usepackage{amssymb,amsmath}
\usepackage{ifxetex,ifluatex}
\ifnum 0\ifxetex 1\fi\ifluatex 1\fi=0 % if pdftex
  \usepackage[T1]{fontenc}
  \usepackage[utf8]{inputenc}
  \usepackage{textcomp} % provides euro and other symbols
\else % if luatex or xelatex
  \usepackage{unicode-math}
  \defaultfontfeatures{Scale=MatchLowercase}
  \defaultfontfeatures[\rmfamily]{Ligatures=TeX,Scale=1}
\fi
% use upquote if available, for straight quotes in verbatim environments
\IfFileExists{upquote.sty}{\usepackage{upquote}}{}
\IfFileExists{microtype.sty}{% use microtype if available
  \usepackage[]{microtype}
  \UseMicrotypeSet[protrusion]{basicmath} % disable protrusion for tt fonts
}{}
\makeatletter
\@ifundefined{KOMAClassName}{% if non-KOMA class
  \IfFileExists{parskip.sty}{%
    \usepackage{parskip}
  }{% else
    \setlength{\parindent}{0pt}
    \setlength{\parskip}{6pt plus 2pt minus 1pt}}
}{% if KOMA class
  \KOMAoptions{parskip=half}}
\makeatother
\usepackage{xcolor}
\IfFileExists{xurl.sty}{\usepackage{xurl}}{} % add URL line breaks if available
\IfFileExists{bookmark.sty}{\usepackage{bookmark}}{\usepackage{hyperref}}
\hypersetup{
  pdftitle={Quantitative Analysis},
  pdfauthor={Matt Clark},
  pdfborder={0 0 0},
  breaklinks=true}
\urlstyle{same}  % don't use monospace font for urls
\usepackage[margin=1in]{geometry}
\usepackage{color}
\usepackage{fancyvrb}
\newcommand{\VerbBar}{|}
\newcommand{\VERB}{\Verb[commandchars=\\\{\}]}
\DefineVerbatimEnvironment{Highlighting}{Verbatim}{commandchars=\\\{\}}
% Add ',fontsize=\small' for more characters per line
\usepackage{framed}
\definecolor{shadecolor}{RGB}{248,248,248}
\newenvironment{Shaded}{\begin{snugshade}}{\end{snugshade}}
\newcommand{\AlertTok}[1]{\textcolor[rgb]{0.94,0.16,0.16}{#1}}
\newcommand{\AnnotationTok}[1]{\textcolor[rgb]{0.56,0.35,0.01}{\textbf{\textit{#1}}}}
\newcommand{\AttributeTok}[1]{\textcolor[rgb]{0.77,0.63,0.00}{#1}}
\newcommand{\BaseNTok}[1]{\textcolor[rgb]{0.00,0.00,0.81}{#1}}
\newcommand{\BuiltInTok}[1]{#1}
\newcommand{\CharTok}[1]{\textcolor[rgb]{0.31,0.60,0.02}{#1}}
\newcommand{\CommentTok}[1]{\textcolor[rgb]{0.56,0.35,0.01}{\textit{#1}}}
\newcommand{\CommentVarTok}[1]{\textcolor[rgb]{0.56,0.35,0.01}{\textbf{\textit{#1}}}}
\newcommand{\ConstantTok}[1]{\textcolor[rgb]{0.00,0.00,0.00}{#1}}
\newcommand{\ControlFlowTok}[1]{\textcolor[rgb]{0.13,0.29,0.53}{\textbf{#1}}}
\newcommand{\DataTypeTok}[1]{\textcolor[rgb]{0.13,0.29,0.53}{#1}}
\newcommand{\DecValTok}[1]{\textcolor[rgb]{0.00,0.00,0.81}{#1}}
\newcommand{\DocumentationTok}[1]{\textcolor[rgb]{0.56,0.35,0.01}{\textbf{\textit{#1}}}}
\newcommand{\ErrorTok}[1]{\textcolor[rgb]{0.64,0.00,0.00}{\textbf{#1}}}
\newcommand{\ExtensionTok}[1]{#1}
\newcommand{\FloatTok}[1]{\textcolor[rgb]{0.00,0.00,0.81}{#1}}
\newcommand{\FunctionTok}[1]{\textcolor[rgb]{0.00,0.00,0.00}{#1}}
\newcommand{\ImportTok}[1]{#1}
\newcommand{\InformationTok}[1]{\textcolor[rgb]{0.56,0.35,0.01}{\textbf{\textit{#1}}}}
\newcommand{\KeywordTok}[1]{\textcolor[rgb]{0.13,0.29,0.53}{\textbf{#1}}}
\newcommand{\NormalTok}[1]{#1}
\newcommand{\OperatorTok}[1]{\textcolor[rgb]{0.81,0.36,0.00}{\textbf{#1}}}
\newcommand{\OtherTok}[1]{\textcolor[rgb]{0.56,0.35,0.01}{#1}}
\newcommand{\PreprocessorTok}[1]{\textcolor[rgb]{0.56,0.35,0.01}{\textit{#1}}}
\newcommand{\RegionMarkerTok}[1]{#1}
\newcommand{\SpecialCharTok}[1]{\textcolor[rgb]{0.00,0.00,0.00}{#1}}
\newcommand{\SpecialStringTok}[1]{\textcolor[rgb]{0.31,0.60,0.02}{#1}}
\newcommand{\StringTok}[1]{\textcolor[rgb]{0.31,0.60,0.02}{#1}}
\newcommand{\VariableTok}[1]{\textcolor[rgb]{0.00,0.00,0.00}{#1}}
\newcommand{\VerbatimStringTok}[1]{\textcolor[rgb]{0.31,0.60,0.02}{#1}}
\newcommand{\WarningTok}[1]{\textcolor[rgb]{0.56,0.35,0.01}{\textbf{\textit{#1}}}}
\usepackage{graphicx,grffile}
\makeatletter
\def\maxwidth{\ifdim\Gin@nat@width>\linewidth\linewidth\else\Gin@nat@width\fi}
\def\maxheight{\ifdim\Gin@nat@height>\textheight\textheight\else\Gin@nat@height\fi}
\makeatother
% Scale images if necessary, so that they will not overflow the page
% margins by default, and it is still possible to overwrite the defaults
% using explicit options in \includegraphics[width, height, ...]{}
\setkeys{Gin}{width=\maxwidth,height=\maxheight,keepaspectratio}
\setlength{\emergencystretch}{3em}  % prevent overfull lines
\providecommand{\tightlist}{%
  \setlength{\itemsep}{0pt}\setlength{\parskip}{0pt}}
\setcounter{secnumdepth}{-2}
% Redefines (sub)paragraphs to behave more like sections
\ifx\paragraph\undefined\else
  \let\oldparagraph\paragraph
  \renewcommand{\paragraph}[1]{\oldparagraph{#1}\mbox{}}
\fi
\ifx\subparagraph\undefined\else
  \let\oldsubparagraph\subparagraph
  \renewcommand{\subparagraph}[1]{\oldsubparagraph{#1}\mbox{}}
\fi

% set default figure placement to htbp
\makeatletter
\def\fps@figure{htbp}
\makeatother

% https://github.com/rstudio/rmarkdown/issues/337
\let\rmarkdownfootnote\footnote%
\def\footnote{\protect\rmarkdownfootnote}

% https://github.com/rstudio/rmarkdown/pull/252
\usepackage{titling}
\setlength{\droptitle}{-2em}

\pretitle{\vspace{\droptitle}\centering\huge}
\posttitle{\par}

\preauthor{\centering\large\emph}
\postauthor{\par}

\predate{\centering\large\emph}
\postdate{\par}

\title{Quantitative Analysis}
\author{Matt Clark}
\date{11/29/2019}

\begin{document}
\maketitle

\hypertarget{introduction}{%
\section{Introduction}\label{introduction}}

R was initially built for analyzing quantitative data. It is still
considered by many people to be the premier platform for scientific
analysis. Analysis may seem a bit scary, but we will see that it's
actually quite easy.

In this lesson, we will go over.

\begin{enumerate}
\def\labelenumi{\arabic{enumi}.}
\tightlist
\item
  Simple T-tests
\item
  ANOVAS
\item
  Single variable linear models
\item
  Multivariat linear models
\item
  Hierarchical linear models
\item
  Uing model objects to predict things and validate models.
\end{enumerate}

\hypertarget{learning-objectives}{%
\section{Learning Objectives}\label{learning-objectives}}

\begin{itemize}
\tightlist
\item
  LO1
\item
  LO2
\item
  LO3
\item
  LO4
\item
  LO6
\end{itemize}

\hypertarget{content}{%
\section{Content}\label{content}}

First let's load the pakages we will use during this lesson

\begin{Shaded}
\begin{Highlighting}[]
\KeywordTok{library}\NormalTok{(tidyverse)}
\end{Highlighting}
\end{Shaded}

\begin{verbatim}
## -- Attaching packages ------------------------------ tidyverse 1.2.1 --
\end{verbatim}

\begin{verbatim}
## v ggplot2 3.2.1     v purrr   0.3.2
## v tibble  2.1.3     v dplyr   0.8.3
## v tidyr   0.8.3     v stringr 1.4.0
## v readr   1.3.1     v forcats 0.4.0
\end{verbatim}

\begin{verbatim}
## -- Conflicts --------------------------------- tidyverse_conflicts() --
## x dplyr::filter() masks stats::filter()
## x dplyr::lag()    masks stats::lag()
\end{verbatim}

\begin{Shaded}
\begin{Highlighting}[]
\KeywordTok{library}\NormalTok{(lme4)}
\end{Highlighting}
\end{Shaded}

\begin{verbatim}
## Loading required package: Matrix
\end{verbatim}

\begin{verbatim}
## 
## Attaching package: 'Matrix'
\end{verbatim}

\begin{verbatim}
## The following object is masked from 'package:tidyr':
## 
##     expand
\end{verbatim}

Now let's load the data Keep in mind that your file path will likely be
different than the file path shown below

\begin{Shaded}
\begin{Highlighting}[]
\NormalTok{dat<-}\KeywordTok{read_csv}\NormalTok{(}\StringTok{"~/Kinesiology_Teaching/Data/Manuscript_Data.csv"}\NormalTok{)}
\end{Highlighting}
\end{Shaded}

\begin{verbatim}
## Parsed with column specification:
## cols(
##   .default = col_double()
## )
\end{verbatim}

\begin{verbatim}
## See spec(...) for full column specifications.
\end{verbatim}

Let's change the Sex variable to be more represintitive of how we
actually think of `Sex' data.

\begin{Shaded}
\begin{Highlighting}[]
\NormalTok{dat}\OperatorTok{$}\NormalTok{Sex<-}\KeywordTok{as.character}\NormalTok{(dat}\OperatorTok{$}\NormalTok{Sex)}

\CommentTok{#now change the vector}
\NormalTok{dat}\OperatorTok{$}\NormalTok{Sex<-dplyr}\OperatorTok{::}\KeywordTok{recode}\NormalTok{(dat}\OperatorTok{$}\NormalTok{Sex, }\StringTok{"1"}\NormalTok{ =}\StringTok{ "Male"}\NormalTok{, }\StringTok{"2"}\NormalTok{ =}\StringTok{ "Female"}\NormalTok{, }\DataTypeTok{.default =} \OtherTok{NA_character_}\NormalTok{)}
\end{Highlighting}
\end{Shaded}

Ok, lets also rename some of the quantitative variables in our dataframe
so that they are easier to deal with

\begin{Shaded}
\begin{Highlighting}[]
\KeywordTok{names}\NormalTok{(dat)[}\DecValTok{52}\NormalTok{]<-}\StringTok{"Variable_1"}
\KeywordTok{names}\NormalTok{(dat)[}\DecValTok{65}\NormalTok{]<-}\StringTok{"Variable_2"}
\KeywordTok{names}\NormalTok{(dat)[}\DecValTok{75}\NormalTok{]<-}\StringTok{"Variable_3"}
\KeywordTok{names}\NormalTok{(dat)[}\DecValTok{100}\NormalTok{]<-}\StringTok{"Variable_4"}
\end{Highlighting}
\end{Shaded}

Let's get rid of the variables we aren't going to use.

There are many ways to do this, here's two (one is commented out):

\begin{Shaded}
\begin{Highlighting}[]
\CommentTok{#dat<-dat%>%select(Subject, Sex, Variable_1,Variable_2,Variable_3,Variable_4)}
\NormalTok{dat<-dat[,}\KeywordTok{c}\NormalTok{(}\DecValTok{1}\NormalTok{,}\DecValTok{2}\NormalTok{,}\DecValTok{52}\NormalTok{,}\DecValTok{65}\NormalTok{,}\DecValTok{75}\NormalTok{,}\DecValTok{100}\NormalTok{)]}
\KeywordTok{head}\NormalTok{(dat)}
\end{Highlighting}
\end{Shaded}

\begin{verbatim}
## # A tibble: 6 x 6
##   Subject Sex    Variable_1 Variable_2 Variable_3 Variable_4
##     <dbl> <chr>       <dbl>      <dbl>      <dbl>      <dbl>
## 1       7 Female       43.5       7.53       35.2       48.5
## 2       8 Male         29.0      12.2        35.2       31.9
## 3       9 Female       51.4      29.6        29.8       56.9
## 4      10 Male         56.9      11.6        32.8       46.9
## 5      11 Male         32.8      15.5        34.1       44.2
## 6      12 Male         30.1      19.6        46.8       47.7
\end{verbatim}

We want to see some statistical significance in our analyses. Let's
order \texttt{Variable\_1}, then create a dummy \texttt{Treatment}
variable that will hopefully look significant.

\begin{Shaded}
\begin{Highlighting}[]
\NormalTok{dat<-}\KeywordTok{arrange}\NormalTok{(dat, Variable_}\DecValTok{1}\NormalTok{)}
\end{Highlighting}
\end{Shaded}

Now let's add the treatment category

\begin{Shaded}
\begin{Highlighting}[]
\NormalTok{dat}\OperatorTok{$}\NormalTok{Treatment<-}\KeywordTok{c}\NormalTok{(}\KeywordTok{rep}\NormalTok{(}\StringTok{"A"}\NormalTok{,}\DecValTok{12}\NormalTok{),}\KeywordTok{rep}\NormalTok{(}\StringTok{"B"}\NormalTok{,}\DecValTok{12}\NormalTok{),}\KeywordTok{rep}\NormalTok{(}\StringTok{"C"}\NormalTok{,}\DecValTok{12}\NormalTok{))}
\end{Highlighting}
\end{Shaded}

Ok, now we should have a treatment variable that may be reflective of
increasing \texttt{Variable\_1}

\#\#\#T-tests Let's compare two things using a t-test

Lets compare treatment A to treatment c

\begin{Shaded}
\begin{Highlighting}[]
\NormalTok{x<-dat}\OperatorTok\KeywordTok{filter}\NormalTok{(Treatment}\OperatorTok{==}\StringTok{"A"}\NormalTok{)}
\NormalTok{y<-dat}\OperatorTok\KeywordTok{filter}\NormalTok{(Treatment}\OperatorTok{==}\StringTok{"C"}\NormalTok{)}
\end{Highlighting}
\end{Shaded}

Now that we have two different dataframes, let's compare them with a
T-test

\begin{Shaded}
\begin{Highlighting}[]
\KeywordTok{t.test}\NormalTok{(x}\OperatorTok{$}\NormalTok{Variable_}\DecValTok{1}\NormalTok{, y}\OperatorTok{$}\NormalTok{Variable_}\DecValTok{1}\NormalTok{)}
\end{Highlighting}
\end{Shaded}

\begin{verbatim}
## 
##  Welch Two Sample t-test
## 
## data:  x$Variable_1 and y$Variable_1
## t = -10.797, df = 20.739, p-value = 5.724e-10
## alternative hypothesis: true difference in means is not equal to 0
## 95 percent confidence interval:
##  -22.30717 -15.09729
## sample estimates:
## mean of x mean of y 
##  36.54730  55.24953
\end{verbatim}

we see here that our p value is very low 5.724e-10, So they are
significantly different.

\#\#\#ANOVAS Now, what if we wanted to compare all three treatments
(A,B,C)

It's always a good isea to plot your data before analyzing it

\begin{Shaded}
\begin{Highlighting}[]
\KeywordTok{ggplot}\NormalTok{(dat, }\KeywordTok{aes}\NormalTok{(}\DataTypeTok{x=}\NormalTok{Treatment, }\DataTypeTok{y=}\NormalTok{Variable_}\DecValTok{1}\NormalTok{, }\DataTypeTok{color=}\NormalTok{Sex))}\OperatorTok{+}
\StringTok{  }\KeywordTok{geom_boxplot}\NormalTok{()}\OperatorTok{+}
\StringTok{  }\KeywordTok{geom_jitter}\NormalTok{()}\OperatorTok{+}
\StringTok{  }\KeywordTok{theme_classic}\NormalTok{()}
\end{Highlighting}
\end{Shaded}

\includegraphics{Lesson4_Quantitative_Analysis_files/figure-latex/unnamed-chunk-10-1.pdf}

It looks like treatment makes a big difference here. Let's run an ANOVA
to compare 3+ groups

\begin{Shaded}
\begin{Highlighting}[]
\NormalTok{fit <-}\StringTok{ }\KeywordTok{aov}\NormalTok{(Variable_}\DecValTok{1} \OperatorTok{~}\StringTok{ }\NormalTok{Treatment, }\DataTypeTok{data=}\NormalTok{dat)}
\KeywordTok{summary}\NormalTok{(fit)}
\end{Highlighting}
\end{Shaded}

\begin{verbatim}
##             Df Sum Sq Mean Sq F value   Pr(>F)    
## Treatment    2 2120.1  1060.0   68.41 1.82e-12 ***
## Residuals   33  511.4    15.5                     
## ---
## Signif. codes:  0 '***' 0.001 '**' 0.01 '*' 0.05 '.' 0.1 ' ' 1
\end{verbatim}

Cool\ldots{}The group's aren't alll the same\ldots{}We probably already
knew that.

There's a phrase that I like that says.

``If there's a real difference, you don't need statistics..this is a
good example.''

What we \emph{can} do though is use a linear model to tell us exactly
what the difference is quantitatively.

\hypertarget{linear-models}{%
\subsubsection{Linear models}\label{linear-models}}

Let's look at the effect that \texttt{Treatment} has on
\texttt{Variable\_1}.

\begin{Shaded}
\begin{Highlighting}[]
\NormalTok{fit <-}\StringTok{ }\KeywordTok{lm}\NormalTok{(Variable_}\DecValTok{1} \OperatorTok{~}\StringTok{ }\NormalTok{Treatment, }\DataTypeTok{data=}\NormalTok{dat)}
\KeywordTok{summary}\NormalTok{(fit)}
\end{Highlighting}
\end{Shaded}

\begin{verbatim}
## 
## Call:
## lm(formula = Variable_1 ~ Treatment, data = dat)
## 
## Residuals:
##     Min      1Q  Median      3Q     Max 
## -7.5593 -3.3471  0.4744  2.2321  7.4651 
## 
## Coefficients:
##             Estimate Std. Error t value Pr(>|t|)    
## (Intercept)   36.547      1.136  32.162  < 2e-16 ***
## TreatmentB    10.988      1.607   6.837 8.40e-08 ***
## TreatmentC    18.702      1.607  11.638 3.19e-13 ***
## ---
## Signif. codes:  0 '***' 0.001 '**' 0.01 '*' 0.05 '.' 0.1 ' ' 1
## 
## Residual standard error: 3.936 on 33 degrees of freedom
## Multiple R-squared:  0.8057, Adjusted R-squared:  0.7939 
## F-statistic: 68.41 on 2 and 33 DF,  p-value: 1.823e-12
\end{verbatim}

The \texttt{lm()} function has a built in \texttt{plot()} capability
that some people think is useful.

\begin{Shaded}
\begin{Highlighting}[]
\KeywordTok{plot}\NormalTok{(fit)}
\end{Highlighting}
\end{Shaded}

\includegraphics{Lesson4_Quantitative_Analysis_files/figure-latex/unnamed-chunk-13-1.pdf}
\includegraphics{Lesson4_Quantitative_Analysis_files/figure-latex/unnamed-chunk-13-2.pdf}

\begin{verbatim}
## hat values (leverages) are all = 0.08333333
##  and there are no factor predictors; no plot no. 5
\end{verbatim}

\includegraphics{Lesson4_Quantitative_Analysis_files/figure-latex/unnamed-chunk-13-3.pdf}
\includegraphics{Lesson4_Quantitative_Analysis_files/figure-latex/unnamed-chunk-13-4.pdf}

Let's also look at our residuals, these should be relatively normal.

\begin{Shaded}
\begin{Highlighting}[]
\KeywordTok{hist}\NormalTok{(}\KeywordTok{residuals}\NormalTok{(fit))}
\end{Highlighting}
\end{Shaded}

\includegraphics{Lesson4_Quantitative_Analysis_files/figure-latex/unnamed-chunk-14-1.pdf}

How do we interpret our model results?

Let's look at \texttt{summary()} again

\begin{Shaded}
\begin{Highlighting}[]
\NormalTok{fit<-}\KeywordTok{lm}\NormalTok{(Variable_}\DecValTok{1}\OperatorTok{~}\NormalTok{Treatment, }\DataTypeTok{data=}\NormalTok{dat)}
\KeywordTok{summary}\NormalTok{(fit)}
\end{Highlighting}
\end{Shaded}

\begin{verbatim}
## 
## Call:
## lm(formula = Variable_1 ~ Treatment, data = dat)
## 
## Residuals:
##     Min      1Q  Median      3Q     Max 
## -7.5593 -3.3471  0.4744  2.2321  7.4651 
## 
## Coefficients:
##             Estimate Std. Error t value Pr(>|t|)    
## (Intercept)   36.547      1.136  32.162  < 2e-16 ***
## TreatmentB    10.988      1.607   6.837 8.40e-08 ***
## TreatmentC    18.702      1.607  11.638 3.19e-13 ***
## ---
## Signif. codes:  0 '***' 0.001 '**' 0.01 '*' 0.05 '.' 0.1 ' ' 1
## 
## Residual standard error: 3.936 on 33 degrees of freedom
## Multiple R-squared:  0.8057, Adjusted R-squared:  0.7939 
## F-statistic: 68.41 on 2 and 33 DF,  p-value: 1.823e-12
\end{verbatim}

Why are there only two treatments displayed?

Treatment\_A is the intercept!

Summary gives us some parameter estimates. What do these mean?

Let's check out the means to get a better idea

\begin{Shaded}
\begin{Highlighting}[]
\KeywordTok{mean}\NormalTok{(dat[dat}\OperatorTok{$}\NormalTok{Treatment}\OperatorTok{==}\StringTok{"A"}\NormalTok{,]}\OperatorTok{$}\NormalTok{Variable_}\DecValTok{1}\NormalTok{)}
\end{Highlighting}
\end{Shaded}

\begin{verbatim}
## [1] 36.5473
\end{verbatim}

\begin{Shaded}
\begin{Highlighting}[]
\KeywordTok{mean}\NormalTok{(dat[dat}\OperatorTok{$}\NormalTok{Treatment}\OperatorTok{==}\StringTok{"B"}\NormalTok{,]}\OperatorTok{$}\NormalTok{Variable_}\DecValTok{1}\NormalTok{)}
\end{Highlighting}
\end{Shaded}

\begin{verbatim}
## [1] 47.53512
\end{verbatim}

\begin{Shaded}
\begin{Highlighting}[]
\KeywordTok{mean}\NormalTok{(dat[dat}\OperatorTok{$}\NormalTok{Treatment}\OperatorTok{==}\StringTok{"C"}\NormalTok{,]}\OperatorTok{$}\NormalTok{Variable_}\DecValTok{1}\NormalTok{)}
\end{Highlighting}
\end{Shaded}

\begin{verbatim}
## [1] 55.24953
\end{verbatim}

The parameter estimate (for a standard linear model) tells us how much
the dependant variable increases as a result of the independant
variables.

Compare the means from above to the parameter estimates from the model.
How are they the same, how are they different?

What about continuous by continuous. Let's plot it.

\begin{Shaded}
\begin{Highlighting}[]
\KeywordTok{ggplot}\NormalTok{(dat, }\KeywordTok{aes}\NormalTok{(}\DataTypeTok{x=}\NormalTok{Variable_}\DecValTok{4}\NormalTok{, }\DataTypeTok{y=}\NormalTok{Variable_}\DecValTok{1}\NormalTok{))}\OperatorTok{+}
\StringTok{  }\KeywordTok{geom_point}\NormalTok{()}\OperatorTok{+}
\StringTok{  }\KeywordTok{geom_smooth}\NormalTok{(}\DataTypeTok{method=}\StringTok{"lm"}\NormalTok{)}\OperatorTok{+}
\StringTok{  }\KeywordTok{theme_classic}\NormalTok{()}
\end{Highlighting}
\end{Shaded}

\includegraphics{Lesson4_Quantitative_Analysis_files/figure-latex/unnamed-chunk-17-1.pdf}

Let's model what we just saw

\begin{Shaded}
\begin{Highlighting}[]
\NormalTok{fit <-}\StringTok{ }\KeywordTok{lm}\NormalTok{(Variable_}\DecValTok{1} \OperatorTok{~}\StringTok{ }\NormalTok{Variable_}\DecValTok{4}\NormalTok{, }\DataTypeTok{data=}\NormalTok{dat)}
\KeywordTok{summary}\NormalTok{(fit)}
\end{Highlighting}
\end{Shaded}

\begin{verbatim}
## 
## Call:
## lm(formula = Variable_1 ~ Variable_4, data = dat)
## 
## Residuals:
##      Min       1Q   Median       3Q      Max 
## -16.2336  -3.5028   0.3521   2.9817  15.9905 
## 
## Coefficients:
##             Estimate Std. Error t value Pr(>|t|)    
## (Intercept)  14.7566     5.7506   2.566   0.0149 *  
## Variable_4    0.6632     0.1183   5.606  2.8e-06 ***
## ---
## Signif. codes:  0 '***' 0.001 '**' 0.01 '*' 0.05 '.' 0.1 ' ' 1
## 
## Residual standard error: 6.342 on 34 degrees of freedom
## Multiple R-squared:  0.4803, Adjusted R-squared:  0.465 
## F-statistic: 31.43 on 1 and 34 DF,  p-value: 2.802e-06
\end{verbatim}

\begin{Shaded}
\begin{Highlighting}[]
\KeywordTok{hist}\NormalTok{(}\KeywordTok{residuals}\NormalTok{(fit))}
\end{Highlighting}
\end{Shaded}

\includegraphics{Lesson4_Quantitative_Analysis_files/figure-latex/unnamed-chunk-18-1.pdf}

\#\#\#Multivariate regression We can look at multivariable regressions
too

Let's look at all of our continuous variables as a predictor for
Variable\_1.

\begin{Shaded}
\begin{Highlighting}[]
\NormalTok{fit<-}\KeywordTok{lm}\NormalTok{(Variable_}\DecValTok{1}\OperatorTok{~}\NormalTok{Variable_}\DecValTok{4}\OperatorTok{+}\NormalTok{Variable_}\DecValTok{3}\OperatorTok{+}\NormalTok{Variable_}\DecValTok{2}\OperatorTok{+}\NormalTok{Treatment, }\DataTypeTok{data=}\NormalTok{dat)}
\KeywordTok{summary}\NormalTok{(fit)}
\end{Highlighting}
\end{Shaded}

\begin{verbatim}
## 
## Call:
## lm(formula = Variable_1 ~ Variable_4 + Variable_3 + Variable_2 + 
##     Treatment, data = dat)
## 
## Residuals:
##     Min      1Q  Median      3Q     Max 
## -6.8312 -3.2235  0.4054  2.5680  7.5444 
## 
## Coefficients:
##             Estimate Std. Error t value Pr(>|t|)    
## (Intercept) 36.45726    6.22756   5.854 2.10e-06 ***
## Variable_4   0.10652    0.12645   0.842 0.406257    
## Variable_3  -0.10610    0.13695  -0.775 0.444557    
## Variable_2  -0.02414    0.10044  -0.240 0.811717    
## TreatmentB   9.31965    2.47048   3.772 0.000711 ***
## TreatmentC  16.63040    2.65675   6.260 6.76e-07 ***
## ---
## Signif. codes:  0 '***' 0.001 '**' 0.01 '*' 0.05 '.' 0.1 ' ' 1
## 
## Residual standard error: 4.037 on 30 degrees of freedom
## Multiple R-squared:  0.8142, Adjusted R-squared:  0.7832 
## F-statistic: 26.29 on 5 and 30 DF,  p-value: 4.018e-10
\end{verbatim}

Note that variable 2 has the highest p value remove that first.

\begin{Shaded}
\begin{Highlighting}[]
\NormalTok{fit<-}\KeywordTok{lm}\NormalTok{(Variable_}\DecValTok{1}\OperatorTok{~}\NormalTok{Variable_}\DecValTok{4}\OperatorTok{+}\NormalTok{Variable_}\DecValTok{3}\OperatorTok{+}\NormalTok{Treatment, }\DataTypeTok{data=}\NormalTok{dat)}
\KeywordTok{summary}\NormalTok{(fit)}
\end{Highlighting}
\end{Shaded}

\begin{verbatim}
## 
## Call:
## lm(formula = Variable_1 ~ Variable_4 + Variable_3 + Treatment, 
##     data = dat)
## 
## Residuals:
##    Min     1Q Median     3Q    Max 
## -6.791 -3.289  0.399  2.544  7.526 
## 
## Coefficients:
##             Estimate Std. Error t value Pr(>|t|)    
## (Intercept) 36.73930    6.02231   6.101 9.22e-07 ***
## Variable_4   0.09897    0.12061   0.821 0.418172    
## Variable_3  -0.11675    0.12759  -0.915 0.367222    
## TreatmentB   9.42803    2.39177   3.942 0.000429 ***
## TreatmentC  16.78712    2.53603   6.619 2.13e-07 ***
## ---
## Signif. codes:  0 '***' 0.001 '**' 0.01 '*' 0.05 '.' 0.1 ' ' 1
## 
## Residual standard error: 3.975 on 31 degrees of freedom
## Multiple R-squared:  0.8138, Adjusted R-squared:  0.7898 
## F-statistic: 33.88 on 4 and 31 DF,  p-value: 6.571e-11
\end{verbatim}

Now note that the p values for the other variables has changed. How have
they changed? They got lower! That's why it's important to remove the
variables one at a time.

\begin{Shaded}
\begin{Highlighting}[]
\NormalTok{fit<-}\KeywordTok{lm}\NormalTok{(Variable_}\DecValTok{1}\OperatorTok{~}\NormalTok{Variable_}\DecValTok{4}\OperatorTok{+}\NormalTok{Treatment, }\DataTypeTok{data=}\NormalTok{dat)}
\KeywordTok{summary}\NormalTok{(fit)}
\end{Highlighting}
\end{Shaded}

\begin{verbatim}
## 
## Call:
## lm(formula = Variable_1 ~ Variable_4 + Treatment, data = dat)
## 
## Residuals:
##     Min      1Q  Median      3Q     Max 
## -7.2672 -3.0126  0.7698  2.2437  7.2071 
## 
## Coefficients:
##             Estimate Std. Error t value Pr(>|t|)    
## (Intercept) 33.27228    4.66922   7.126 4.37e-08 ***
## Variable_4   0.08648    0.11953   0.723 0.474638    
## TreatmentB   9.74482    2.36056   4.128 0.000244 ***
## TreatmentC  17.37438    2.44724   7.100 4.70e-08 ***
## ---
## Signif. codes:  0 '***' 0.001 '**' 0.01 '*' 0.05 '.' 0.1 ' ' 1
## 
## Residual standard error: 3.965 on 32 degrees of freedom
## Multiple R-squared:  0.8088, Adjusted R-squared:  0.7909 
## F-statistic: 45.12 on 3 and 32 DF,  p-value: 1.334e-11
\end{verbatim}

We see now that none of our continuous variables are important when we
account for Treatment

They are whe we only usse Variable\_4 though..This is a good lesson in
model selection!

Now control for sex. You can make a 3d plot or a density plot to plot 2
variable regressions.

We're not going to show these.

Let's do the same type of multivariate regression as before.

\begin{Shaded}
\begin{Highlighting}[]
\NormalTok{fit<-}\KeywordTok{lm}\NormalTok{(Variable_}\DecValTok{1}\OperatorTok{~}\NormalTok{Sex}\OperatorTok{+}\NormalTok{Treatment, }\DataTypeTok{data=}\NormalTok{dat)}
\KeywordTok{summary}\NormalTok{(fit)}
\end{Highlighting}
\end{Shaded}

\begin{verbatim}
## 
## Call:
## lm(formula = Variable_1 ~ Sex + Treatment, data = dat)
## 
## Residuals:
##     Min      1Q  Median      3Q     Max 
## -6.6966 -2.8492  0.7807  2.6856  8.5435 
## 
## Coefficients:
##             Estimate Std. Error t value Pr(>|t|)    
## (Intercept)   38.273      1.388  27.579  < 2e-16 ***
## SexMale       -2.588      1.293  -2.002   0.0538 .  
## TreatmentB    10.341      1.572   6.578 2.06e-07 ***
## TreatmentC    18.487      1.542  11.987 2.26e-13 ***
## ---
## Signif. codes:  0 '***' 0.001 '**' 0.01 '*' 0.05 '.' 0.1 ' ' 1
## 
## Residual standard error: 3.768 on 32 degrees of freedom
## Multiple R-squared:  0.8273, Adjusted R-squared:  0.8111 
## F-statistic:  51.1 on 3 and 32 DF,  p-value: 2.644e-12
\end{verbatim}

A better way to do this is to give each sex its own intercept, slope, or
both. These will both be more representitive of reality.

Let's plot both

\begin{Shaded}
\begin{Highlighting}[]
\KeywordTok{ggplot}\NormalTok{(dat, }\KeywordTok{aes}\NormalTok{(}\DataTypeTok{x=}\NormalTok{Variable_}\DecValTok{4}\NormalTok{, }\DataTypeTok{y=}\NormalTok{Variable_}\DecValTok{1}\NormalTok{, }\DataTypeTok{color=}\NormalTok{Sex))}\OperatorTok{+}
\StringTok{  }\KeywordTok{geom_point}\NormalTok{()}\OperatorTok{+}
\StringTok{  }\KeywordTok{geom_smooth}\NormalTok{(}\DataTypeTok{method=}\StringTok{"lm"}\NormalTok{)}\OperatorTok{+}
\StringTok{  }\KeywordTok{theme_classic}\NormalTok{()}
\end{Highlighting}
\end{Shaded}

\includegraphics{Lesson4_Quantitative_Analysis_files/figure-latex/unnamed-chunk-23-1.pdf}

We need a new package to fit this model.

\begin{Shaded}
\begin{Highlighting}[]
\NormalTok{fit <-}\StringTok{ }\KeywordTok{lmer}\NormalTok{(Variable_}\DecValTok{1} \OperatorTok{~}\StringTok{ }\NormalTok{Variable_}\DecValTok{4}\OperatorTok{+}\NormalTok{(Variable_}\DecValTok{4}\OperatorTok{|}\NormalTok{Sex), }\DataTypeTok{data=}\NormalTok{dat)}
\end{Highlighting}
\end{Shaded}

\begin{verbatim}
## boundary (singular) fit: see ?isSingular
\end{verbatim}

\begin{Shaded}
\begin{Highlighting}[]
\KeywordTok{summary}\NormalTok{(fit)}
\end{Highlighting}
\end{Shaded}

\begin{verbatim}
## Linear mixed model fit by REML ['lmerMod']
## Formula: Variable_1 ~ Variable_4 + (Variable_4 | Sex)
##    Data: dat
## 
## REML criterion at convergence: 233.6
## 
## Scaled residuals: 
##      Min       1Q   Median       3Q      Max 
## -2.53982 -0.59301  0.02442  0.44377  2.48968 
## 
## Random effects:
##  Groups   Name        Variance  Std.Dev. Corr 
##  Sex      (Intercept) 2.007e+00 1.41671       
##           Variable_4  2.374e-04 0.01541  -1.00
##  Residual             3.996e+01 6.32125       
## Number of obs: 36, groups:  Sex, 2
## 
## Fixed effects:
##             Estimate Std. Error t value
## (Intercept)  15.0182     5.8450   2.569
## Variable_4    0.6583     0.1189   5.538
## 
## Correlation of Fixed Effects:
##            (Intr)
## Variable_4 -0.980
## convergence code: 0
## boundary (singular) fit: see ?isSingular
\end{verbatim}

If we wanted each sex to just have its own intercept.

\begin{Shaded}
\begin{Highlighting}[]
\NormalTok{fit <-}\StringTok{ }\KeywordTok{lmer}\NormalTok{(Variable_}\DecValTok{1} \OperatorTok{~}\StringTok{ }\NormalTok{Variable_}\DecValTok{4}\OperatorTok{+}\NormalTok{(}\DecValTok{1}\OperatorTok{|}\NormalTok{Sex), }\DataTypeTok{data=}\NormalTok{dat)}
\KeywordTok{summary}\NormalTok{(fit)}
\end{Highlighting}
\end{Shaded}

\begin{verbatim}
## Linear mixed model fit by REML ['lmerMod']
## Formula: Variable_1 ~ Variable_4 + (1 | Sex)
##    Data: dat
## 
## REML criterion at convergence: 233.6
## 
## Scaled residuals: 
##     Min      1Q  Median      3Q     Max 
## -2.5416 -0.5857  0.0297  0.4533  2.4962 
## 
## Random effects:
##  Groups   Name        Variance Std.Dev.
##  Sex      (Intercept)  0.4124  0.6421  
##  Residual             40.0085  6.3252  
## Number of obs: 36, groups:  Sex, 2
## 
## Fixed effects:
##             Estimate Std. Error t value
## (Intercept)  14.8987     5.7631   2.585
## Variable_4    0.6606     0.1181   5.592
## 
## Correlation of Fixed Effects:
##            (Intr)
## Variable_4 -0.980
\end{verbatim}

{[}\url{http://mfviz.com/hierarchical-models/}{]}{[}Here is a great
resource explaining heierarchical modeling.{]}

\hypertarget{predictions-and-model-validation}{%
\subsubsection{Predictions and model
validation}\label{predictions-and-model-validation}}

Making predictions is a really powerful. We like to be able to predict
things, but also to gutcheck yourself!

Make a new dataframe with some hypothetical values to put into the model

\begin{Shaded}
\begin{Highlighting}[]
\NormalTok{new <-}\StringTok{ }\KeywordTok{data.frame}\NormalTok{(}\DataTypeTok{Variable_4 =} \DecValTok{25}\NormalTok{, }\DataTypeTok{Treatment=}\StringTok{"B"}\NormalTok{, }\DataTypeTok{Sex=}\StringTok{"Male"}\NormalTok{)}
\end{Highlighting}
\end{Shaded}

Now put it into a predict function

\begin{Shaded}
\begin{Highlighting}[]
\KeywordTok{predict}\NormalTok{(fit, new)}
\end{Highlighting}
\end{Shaded}

\begin{verbatim}
##        1 
## 31.23722
\end{verbatim}

\end{document}
